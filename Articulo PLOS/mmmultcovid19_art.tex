% Options for packages loaded elsewhere
\PassOptionsToPackage{unicode}{hyperref}
\PassOptionsToPackage{hyphens}{url}
%
\documentclass[
]{article}
\usepackage{lmodern}
\usepackage{amssymb,amsmath}
\usepackage{ifxetex,ifluatex}
\ifnum 0\ifxetex 1\fi\ifluatex 1\fi=0 % if pdftex
  \usepackage[T1]{fontenc}
  \usepackage[utf8]{inputenc}
  \usepackage{textcomp} % provide euro and other symbols
\else % if luatex or xetex
  \usepackage{unicode-math}
  \defaultfontfeatures{Scale=MatchLowercase}
  \defaultfontfeatures[\rmfamily]{Ligatures=TeX,Scale=1}
\fi
% Use upquote if available, for straight quotes in verbatim environments
\IfFileExists{upquote.sty}{\usepackage{upquote}}{}
\IfFileExists{microtype.sty}{% use microtype if available
  \usepackage[]{microtype}
  \UseMicrotypeSet[protrusion]{basicmath} % disable protrusion for tt fonts
}{}
\makeatletter
\@ifundefined{KOMAClassName}{% if non-KOMA class
  \IfFileExists{parskip.sty}{%
    \usepackage{parskip}
  }{% else
    \setlength{\parindent}{0pt}
    \setlength{\parskip}{6pt plus 2pt minus 1pt}}
}{% if KOMA class
  \KOMAoptions{parskip=half}}
\makeatother
\usepackage{xcolor}
\IfFileExists{xurl.sty}{\usepackage{xurl}}{} % add URL line breaks if available
\IfFileExists{bookmark.sty}{\usepackage{bookmark}}{\usepackage{hyperref}}
\hypersetup{
  pdftitle={Multi-level multi-state modelling applied to hospital admission in mexican patients with COVID-19},
  hidelinks,
  pdfcreator={LaTeX via pandoc}}
\urlstyle{same} % disable monospaced font for URLs
\usepackage[margin=1in]{geometry}
\usepackage{graphicx}
\makeatletter
\def\maxwidth{\ifdim\Gin@nat@width>\linewidth\linewidth\else\Gin@nat@width\fi}
\def\maxheight{\ifdim\Gin@nat@height>\textheight\textheight\else\Gin@nat@height\fi}
\makeatother
% Scale images if necessary, so that they will not overflow the page
% margins by default, and it is still possible to overwrite the defaults
% using explicit options in \includegraphics[width, height, ...]{}
\setkeys{Gin}{width=\maxwidth,height=\maxheight,keepaspectratio}
% Set default figure placement to htbp
\makeatletter
\def\fps@figure{htbp}
\makeatother
\setlength{\emergencystretch}{3em} % prevent overfull lines
\providecommand{\tightlist}{%
  \setlength{\itemsep}{0pt}\setlength{\parskip}{0pt}}
\setcounter{secnumdepth}{-\maxdimen} % remove section numbering
\ifluatex
  \usepackage{selnolig}  % disable illegal ligatures
\fi

\title{Multi-level multi-state modelling applied to hospital admission
in mexican patients with COVID-19}
\author{true \and true \and true \and true}
\date{}

\begin{document}
\maketitle
\begin{abstract}
Lorem ipsum dolor sit amet, consectetur adipiscing elit. Curabitur eget
porta erat. Morbi consectetur est vel gravida pretium. Suspendisse ut
dui eu ante cursus gravida non sed sem. Nullam sapien tellus, commodo id
velit id, eleifend volutpat quam. Phasellus mauris velit, dapibus
finibus elementum vel, pulvinar non tellus. Nunc pellentesque pretium
diam, quis maximus dolor faucibus id. Nunc convallis sodales ante, ut
ullamcorper est egestas vitae. Nam sit amet enim ultrices, ultrices elit
pulvinar, volutpat risus.
\end{abstract}

\begin{quote}
\begin{quote}
\begin{quote}
\begin{quote}
\begin{quote}
\begin{quote}
\begin{quote}
Stashed changes \newcommand{\N}{\mathbb{N}} \newcommand{\Z}{\mathbb{Z}}
\newcommand{\R}{\mathbb{R}} \newcommand{\Q}{\mathbb{Q}}
\newcommand{\vac}{\varnothing} \newcommand{\Pro}{\mathbb{P}}
\newcommand{\var}{\text{Var}} \newcommand{\E}{\mathbb{E}}
\end{quote}
\end{quote}
\end{quote}
\end{quote}
\end{quote}
\end{quote}
\end{quote}

\emph{Text based on plos sample manuscript, see
\url{https://journals.plos.org/ploscompbiol/s/latex}}

\hypertarget{introduction}{%
\subsection{Introduction}\label{introduction}}

The SARS-CoV-2 pandemic was declared a Public Health Emergency of
International Concern on January 30, 2020 by the World Health
Organization.The Mexican Health Authorithies declared the first lockdown
on March 26 with 585 cases and 8 deaths reported for COVID-19 (2); at
the end of the lockdown (june 5th 2020) the total cases were 110,026 and
13 170 deaths. Until November 1, Mexico is the fourth country in death
rates of SARS-CoV-19 (106,765 deaths), with 1,122,362 incident cases
(3).

Over time it has become clear that the presence of comorbidities such as
hypertension, diabetes, obesity and smoking are factors that increase
the serious illness that leads to hospitalization and in 25\% of the
cases they required admission and intubation to the intensive care unit
(4). Mexico ranks second in obesity among OECD countries, with almost
72.5\% obesity among the adult population, which is associated with the
high prevalence of type 2 diabetes, estimated at 13\% of the adult
population in 2017, which is the highest rate among OECD countries (5);
hypertension is also one of the higher chronic diseases among adult
population 30\% (6). The high prevalence of this comorbidities besides
the lack of a functional health care system is believed to be the main
reason why the severe cases and deaths rates in the country are so high.

After onset of infection there is a period of time between symptom
detection and hospitalization. The time elapsed before patients approach
hospitals could be excessively long. Once patients are admmited to
hospital, there is also a period of time between the admission and
death. Estimation of this times through a multilevel model could enable
a better information system to estimate incidence and transmission
rates, particularly at regional level since differences have already
been established.

Data used in the modeling comes from the official database by the
Mexican Ministry of Health; the analysis provides of general overview of
hospitalizations in each state of the country and the different health
institutions within. This work considers a multi-state model to describe
hospital resource usage enabling an evaluation of disease severity.

This study shows how different the outcomes can be for the patient due
the late hospitalization, this lead to believe that the average patients
waits until the symptoms are severe to seek professional healthcare.

Here are two sample references: @Feynman1963118 {[}@Dirac1953888{]}.

\hypertarget{methods-and-materials}{%
\subsection{Methods and materials}\label{methods-and-materials}}

\hypertarget{data-source-and-study-population}{%
\subsubsection{Data Source and Study
Population}\label{data-source-and-study-population}}

\textless\textless\textless\textless\textless\textless\textless{}
Updated upstream We conducted a prevalence study the official database
from the Mexican Ministry of Health, this data provides a overview of
hospital admissions, deaths and the period of time between
hospitalizations and first symptoms between March and XX 2020. The data
analyzed included mexican adult population diagnosed with COVID-19 in
the whole country; the exclusion criteria were the observations with
incomplete data about hospital admission, symptoms or comorbidities.
After applying exclusion criteria the total sample was about XXXX
registers of adult patients belonging to any healthcare institution,
either private or public in the 32 states of Mexico.

Comorbidities that could worsen the patient outcome such as diabetes,
hypertension, obesity, chronic obstructive pulmonary disease (COPD),
asthma, autoimmune diseases and chronic kidney disease were considered
and also the interactions between two or more was simulated.

We conducted a prevalence study the official database from the Mexican
Ministry of Health, this data provides a overview of hospital
admissions, deaths and the period of time between hospitalizations and
first symptoms between March and December 2020. The data analyzed
included mexican adult population diagnosed with COVID-19 in the whole
country; the exclusion criteria were the observations with incomplete
data about hospital admission, symptoms or comorbidities. Additionally
patients whose time of initial symptoms was captured as the day the were
admitted to hospital were removed, since this time was likely to be
unknown. After applying exclusion criteria a total sample of 1200
registers of adult patients belonging to any healthcare institution,
either private or public in the 32 states of Mexico was selected,
preserving the population characteristics. JUAN PABLO ESTE NUMERO DE
MUESTRA ES CORRECTO?

Comorbidities that could worsen the patient outcome such as diabetes,
hypertension, obesity, chronic obstructive pulmonary disease (COPD),
asthma, immunosuppression and chronic kidney disease were cataloged for
each health state in the model and so two different matrix were set up
as it follows:

\[x_{death} = (\sim Diabetes+COPD+Obesity+Hypertension+Diabetes*Obesity*Hypertension+Kidney\_Disease)\]
\[x_{hospitalization} = (\sim COPD+Obesity+Kidney\_Disease+Asthma+Immunosuppression)
\]

About 87\% of population in Mexico belong to some healthcare institution
but during this pandemic mexican government has established a list of
hospitals designated to treat COVID-19 patients without any affiliation
distinction. In this study we identified 6 different healthcare
providers which were classified as sectors IMSS, ISSSTE,
SEDENA/SEMAR/PEMEX, SSA, ESTATALES (healthcare provider within each
state) this 5 are the public care provider and the sixth sector is
private hospitals.

\hypertarget{modelling}{%
\subsubsection{Modelling}\label{modelling}}

We developed four different Bayesian models for trajectories of interest
namely, \emph{Symptoms-Hospitalization} and
\emph{Hospitalization-Death}, in which non-informative initial
distributions were used, located near \(0\), to improve convergence.
Additionally a \textit{QR} reparameterization for the covariable matrix
was used, that is, if \(X\) is an \(n \times m\) covariable matrix,
corresponding to the aforementioned comorbidities, \(X=QR\), where \(Q\)
is an ortogonal matrix and \(R\) is an upper triangular matrix. In
practice, considering \(X=Q'R'\) where \(Q'=Q\sqrt{n-1}\) and
\(R'=\frac{1}{\sqrt{n-1}}R\) is convinient. Hence if \(\zeta\) is the
\textit{N} linear predictor vector such that \(\zeta=X\beta\), with
\(\beta\) a \textit{K} coefficient vector, then
\(\zeta=X\beta=QR\beta=Q'R'\beta\). We used \(\zeta=Q'R'\beta\) for
numerical stability.

Each model has different levels on it,as more levels were included it
was possible to see how different the results were according to the new
variable added.The four levels of information were: 1. Data of death and
hospitalizations which are assumed independent. 2. This model is based
in the first model, but here the information is slipt up by the
different states of Mexico. 3. Third model takes the information of the
second one and adds the type/institution in which the patients were
treated. 4. The last models adds the information to the third one by
each state of the Mexican republic and so all the variables are
accounted in the last one.

To choose the model that will describe this analysis, a goodness-of-fit
was applied to each one of the models, these adjustments are described
in table XX.

\hypertarget{model-i-one-level}{%
\paragraph{Model I: One level}\label{model-i-one-level}}

Patient \(i\) corresponds to the \(i\)-th row of vectors \(M\) and \(H\)
for deaths and hospitalizations, respectively. We considered a different
set of covariables in each case and we assumed deaths and
hospitalizations are independent and the model is given as

\[
\begin{aligned}
 {M}  &\sim Weibull(\alpha,\mathbf{\eta})\\
 {H}  &\sim Weibull(\alpha,\mathbf{\upsilon}) \\
 \mathbf{\eta} &= \exp\left(-\frac{\mu_m+\mathbf{Q}^*\mathbf{\vartheta}}{\alpha}\right) \\
 \mathbf{\upsilon} &= \exp\left(-\frac{\mu_h+\mathbf{Q}^{**}\mathbf{\theta}}{\alpha}\right) \\
 \alpha&=\exp(\alpha_r*\tau_\alpha) \\
 \alpha_r&\sim N(0,1) \\
 \mu_m,\mu_h &\sim N(0,\tau_\mu) \\
 \mathbf{\vartheta},\mathbf{\theta} &\sim U(-\infty,\infty) \\
\end{aligned}
\]

where \(Q^*\) and \(Q^{**}\) are matrices of standarized covariables for
deaths and hospitalizations respectively and \(\tau_\alpha\) and
\(\tau_{\mu}\) are given positive values. This model is described in
grey in Figure REF.

\hypertarget{model-ii-two-levels}{%
\paragraph{Model II: two levels}\label{model-ii-two-levels}}

The second model is based on the first one, an additional level is added
to account for each state of Mexico to model deaths. The hospitalization
\(H\) remains unchanged and for each state \(l= 1,\ldots, 32\) and
patient \(i\), deaths are summarized in matriz \(M_{l}\) the model is
defined as

\[
\begin{aligned}
 M_l   &\sim Weibull(\alpha,\mathbf{\eta})\\
 H  &\sim Weibull(\alpha,\mathbf{\upsilon}) \\
 \mathbf{\eta} &= \exp\left(-\frac{\mu_m+\mathbf{\mu}_l^r+\mathbf{Q}^*\mathbf{\vartheta}}{\alpha}\right),\space\space l=1,...,32\\
 \mathbf{\upsilon} &= \exp(-\frac{\mu_h+\mathbf{Q}^{**}\mathbf{\theta}}{\alpha}) \\
 \mathbf{\mu}_l&=\sigma*\mathbf{\mu}_l^r \\
 \alpha&=\exp(\alpha^r*\tau_\alpha) \\
 \alpha^r&\sim N(0,1) \\
 \mu^r_l&\sim N(0,1) \\
 \sigma&\sim t^+_3(0,1) \\
 \mu_m,\mu_h &\sim N(0,\tau_\mu) \\
 \mathbf{\vartheta},\mathbf{\theta} &\sim U(-\infty,\infty) \\
\end{aligned}
\] where \(Q^*\) and \(Q^{**}\) are matrices of standarized covariables
for deaths and hospitalizations respectively and \(\tau_\alpha\) and
\(\tau_{\mu}\) are given positive values. This model is described in
green in Figure REF.

\hypertarget{model-iii-three-levels}{%
\paragraph{Model III: Three levels}\label{model-iii-three-levels}}

Based on Model II, we consider a third level to include the type of
health service where patients are hospitalized, \(k\), for patient \(i\)
in state \(l\) we have \(M_{l,k}\) for the corresponding death matrix
and the model is given as

\[
\begin{aligned}
 M_{l,k}   &\sim Weibull(\alpha,\mathbf{\eta})\\
 H  &\sim Weibull(\alpha,\mathbf{\upsilon}) \\
 \mathbf{\eta} &= \exp \left(-\frac{\mu_m+\mathbf{\mu}_l^r+\mathbf{\mu}_k^r+\mathbf{Q}^*\mathbf{\vartheta}}{\alpha} \right),\space\space l=1,...,32,\space k=1,...,5\\
 \mathbf{\upsilon} &= \exp\left(-\frac{\mu_h+\mathbf{Q}^{**}\mathbf{\theta}}{\alpha}\right) \\
  \mathbf{\mu}_l&=\sigma_l*\mathbf{\mu}_l^r, \space\space l=1,2 \\
 \mathbf{\mu}_k&=\sigma_l*\mathbf{\mu}_k^r \\
 \alpha&=\exp(\alpha^r*\tau_\alpha) \\
 \alpha^r&\sim N(0,1) \\
 \mu^r_l,\mu^r_k &\sim N(0,1) \\
 \sigma_l&\sim t^+_3(0,1) \\
 \mu_m,\mu_h &\sim N(0,\tau_\mu) \\
 \mathbf{\vartheta},\mathbf{\theta} &\sim U(-\infty,\infty) \\
\end{aligned}
\] where \(Q^*\) and \(Q^{**}\) are matrices of standarized covariables
for deaths and hospitalizations respectively and \(\tau_\alpha\) and
\(\tau_{\mu}\) are given positive values. This model is described in
yellow in Figure REF.

\hypertarget{model-iv}{%
\paragraph{Model IV:}\label{model-iv}}

Based as well on Modelo II, however, we consider index \(j=(l,k)\) where
\(l\) accounts for the \(l\)-th state and \(k\) for the type of health
service, \(j \in \{1,...,153\}\) where the distribution for deaths is
given by

\[M_j \sim Weibull\left(\alpha, exp\left(-\frac{Q^{**}\theta^m+\mu_k+\mu_l}{\alpha} \right) \right)\]

\hypertarget{tabla-ejemplo}{%
\section{Tabla ejemplo}\label{tabla-ejemplo}}

\hypertarget{results}{%
\section{Results}\label{results}}

We show results for model four which performed better, both in
convergence and goodness of fit criteria (¿PODRIAMOS AGREGAR QUIZÁ LA
MAXIMA ESTADÍSTICA DE CONVERGENCIA O EL PROMEDIO PARA LOS PARAMETROS DEL
MODELO O DE TODOS LOS MODELOS?) . We show the posterior 0.95 credibility
intervals for parameters of interest at different levels of the model.
It is worth pointing out that we are displaying the log hazard ratio,
hence positive values for parameters will point to increasing risks for
the corresponding transition and level.

Increased risk for hospitalization was observed at the global population
level for chronic renal disease, whereas for death such was the case for
COPD and the interaction of diabetes:hypertension:obesity.

This study has shown that there are differences in mortality between the
states without accounting for institution, if \(\beta>0\) the risk
increases whereas \(\beta<0\) decreases the risk of the event, and it is
related to the prompt time of death or viceversa. Image Mu\_1Jer1 shows
the states in which the overall rate of mortality is higher possibly due
the late hospitalization of patients such as Campeche, Colima,
Guanajuato, Hidalgo, Jalisco, Morelos, Nayarit, Oaxaca, Puebla, Tabasco
and Veracruz.

\hypertarget{disscusion}{%
\section{Disscusion}\label{disscusion}}

Through time it has been proved that the presence of comorbidities such
as diabetes, hypertension, obesity, chronic obstructive pulmonary
disease (COPD), asthma, inmmuno-suppression and chronic kidney disease
are associated to a worse outcome for the patient diagnosed with
COVID-19, specially those who are hospitalized and ICU due intubation;
but to our knowledge this is the first study who analyzed the time
elapsed between the patient first symptoms, hospitalization and death;
all this analysis can break down to the different states of the republic
and the healthcare institution in them. Because of this it is possible
to distinguish those for which the risk of hospitalization and death
increases.

This analysis can be helpful to a regional level to improve healthcare
assistance, additionally it could be useful to inform statistical
estimation of parameters for an epidemiological model ADD REFERENCE.

One of the main problems is the precarious situation of the public
healthcare system which universal coverage is estimated about 87\% of
the mexican population. It´s clear that mexican healthcare has overrun
during this pandemic and has appeal to private health providers to cope
with the treatment of COVID-19 patients. Mexican population is the third
place in obesity among the OCDE countries such thing as this one is one
of the main reason why the severe cases due COVID-19 are so high.

This study has show that the differences in mortality between the
different states of the republic; there are those states in which the
overall rate of mortality is higher due the late hospitalization of
patients such as Veracruz, Nuevo Leon, San Luis Potosí, Guanajuato,
Chiapas and Mexico City. Breaking down this analysis to level state we
can deduce a higher risk of hospitalization, specifically in Veracruz HR
which historically has been unsteady regarding the public healthcare
system in sectors like IMSS, ISSSTE, SEDENA/SEMAR/PEMEX.

This analysis can be helpful to a regional level to improve healthcare
assistance, additionally it could be useful to inform statistical
estimation of parameters for an epidemiological model ADD REFERENCE.

One of the main limitations was due the lack of hospitalization

\hypertarget{references}{%
\section*{References}\label{references}}
\addcontentsline{toc}{section}{References}

\end{document}
