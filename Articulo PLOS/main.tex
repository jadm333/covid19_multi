\documentclass{article}
\usepackage[utf8]{inputenc}
\usepackage{verbatim}
\usepackage[spanish,es-noshorthands]{babel}
\usepackage{tikz,times}
\usepackage{amsmath}
\usepackage{amsfonts}
\usepackage{amssymb}
\usepackage{float}
\usepackage{dsfont}

\usepackage{enumerate}
\usepackage{lastpage}
%permite detectar la última pagina para usarla en la cabecera o en el pie de pagina

\usepackage{listings}

\usepackage{chngcntr}   

\usepackage{etoolbox}
\usepackage{lscape}
\usepackage{bbold}
\usepackage{graphicx}



\usepackage[left=2.5cm,right=2.5cm,top=2.6cm,bottom=2.6cm]{geometry}

\newcommand{\N}{\mathbb{N}}
\newcommand{\Z}{\mathbb{Z}}
\newcommand{\R}{\mathbb{R}}
\newcommand{\Q}{\mathbb{Q}}
\newcommand{\vac}{\varnothing}
\newcommand{\Pro}{\mathbb{P}}
\newcommand{\var}{\text{Var}}
\newcommand{\E}{\mathbb{E}}



\begin{document}





\section{Model}

We considered four different Bayesian models  for trajectories of interest:  \textit{Symptoms-Hospitalization}, \textit{Hospitalization-Death}, in which non-informative initial distributions were used, located near $0$, to improve convergence. Additionally a  \textit{QR} reparameterization for the covariable matrix was used, that is, if  $X$ is an  $N \times K$ covariable matrix, $X=QR$, where $Q$ is an ortogonal  matrix and  $R$ is an upper triangular matriz. In practice,  considering $X=Q'R'$ where $Q'=Q*\sqrt{n-1}$ and $R'=\frac{1}{\sqrt{n-1}}*R$ is convinient. Hence if  $\zeta$ is the  \textit{N} linear predictor  vector such that $\zeta=X\beta$, with $\beta$ a \textit{K}  coefficient vector, then  $\zeta=X\beta=QR\beta=Q'R'\beta$. We used  $\zeta=Q'R'\beta$  for numerical stability. 


\textbf{Modelo I: One level}

For patient  $i$, with $M$ covariables, we assumed deaths and hospitalizations, $M_i$ y $H_i$ respectively,  are independent and follow  distributions: 
\begin{center}
    \begin{equation}\nonumber
        M_i \sim Weibull(\alpha, exp\{-(\frac{(Q^m\theta^m)+\mu^m}{\alpha})\}) \:\:\:\:\:\: y \:\:\: \:\:\:  H_i \sim Weibull(\alpha, exp\{-(\frac{(Q^h\theta^h)+\mu^h}{\alpha})\})
    \end{equation}
\end{center}

where  
\begin{itemize}
    \item $\alpha=\tau_\alpha \beta$ for $\tau_\alpha$ a given positive value and $\beta\sim Normal(0,1)$.\\
    \item $\theta^m$ and $\theta^h$ are $M$-dimensional vectors.\\
    \item $Q^m$ and $Q^h$ are matrices of standarized covariables for deaths and hospitalizations respectively.\\
    \item Hyperparameters $\mu ^{m}$ and $\mu ^{h}$, for deaths and hospitalizations correspondingly, with a given $\tau_{\mu}$ follow a normal distribution
    
       \begin{center}
          \begin{equation} \nonumber 
              \mu^m, \mu ^h \sim Normal (0, \tau_{\mu}) \\
          \end{equation}
       \end{center}
       
  \end{itemize}

\textbf{Model II: two levels}
Using the first model an additional level is added to account for each state of Mexico to model deaths. The hospitalization  $H_i$  remains  unchanged and for each state  $j= 1,\ldots, 32$ and  patient $i$,  $M_{i,j}$ has distribution:

\begin{center}
    \begin{equation}\nonumber
        M_{i,j} \sim Weibull(\alpha, exp\{- \frac{(Q^m\theta^m)+\mu^m+\mu^{L_1}}{\alpha} \})\\
    \end{equation}
\end{center}

Where  $\mu^{L_1}=\nu^{L_1}*\frac{\sigma^{L_1}}{\tau^{L_1}}$, have the following distributions
\begin{itemize}
     
    \item $\nu^{L1} \sim Normal(0,1)$
    \item $\sigma^{L1} \sim Normal(0,1)$
    \item $\tau^{L1} \sim Gamma(a,a)$ with  $a0$ a given positive value.
\end{itemize}



\subsection{Model III: Three levels}
Based on  Model II, we consider a third level to include the type of health service where patients are hospitalized, $k$ for patient
 $i$ in state  $j$ we have $M_{i,j,k}$ distributed as


\begin{center}
    \begin{equation}\nonumber
        M_{i,j,k} \sim Weibull(\alpha, exp\{- \frac{(Q^m\theta^m)+\mu^m+\mu^{L_1}+\mu^{L_2}}{\alpha} \})
    \end{equation}
\end{center}
Where $\mu^{L_2}=\nu^{L_2}*\frac{\sigma^{L_2}}{\tau^{L_2}}$, with distributions: 
\begin{itemize}
     
    \item $\nu^{L_2} \sim Normal(0,1)$
    \item $\sigma^{L_2} \sim Normal(0,1)$
    \item $\tau^{L_2} \sim Gamma(a,a)$ with $a$ a given positive value.
\end{itemize}

\subsection{Model IV:}
Based as well in  \textit{Modelo II}, however, we consider index� $l=(j,k)$ where $j$ accounts for the $j-$-th state  and $k$ for the type of health service, �$j \in \{1,...,153\}$ where the distribution for deaths is given by

\begin{center}
    \begin{equation}\nonumber
        M_{i,l} \sim Weibull(\alpha, exp\{- \frac{(Q^m\theta^m)+\mu^m+\mu^{L_1}}{\alpha} \})\\
    \end{equation}
\end{center}

\end{document}
