% Template for PLoS
% Version 3.5 March 2018
%
% % % % % % % % % % % % % % % % % % % % % %
%
% -- IMPORTANT NOTE
%
% This template contains comments intended
% to minimize problems and delays during our production
% process. Please follow the template instructions
% whenever possible.
%
% % % % % % % % % % % % % % % % % % % % % % %
%
% Once your paper is accepted for publication,
% PLEASE REMOVE ALL TRACKED CHANGES in this file
% and leave only the final text of your manuscript.
% PLOS recommends the use of latexdiff to track changes during review, as this will help to maintain a clean tex file.
% Visit https://www.ctan.org/pkg/latexdiff?lang=en for info or contact us at latex@plos.org.
%
%
% There are no restrictions on package use within the LaTeX files except that
% no packages listed in the template may be deleted.
%
% Please do not include colors or graphics in the text.
%
% The manuscript LaTeX source should be contained within a single file (do not use \input, \externaldocument, or similar commands).
%
% % % % % % % % % % % % % % % % % % % % % % %
%
% -- FIGURES AND TABLES
%
% Please include tables/figure captions directly after the paragraph where they are first cited in the text.
%
% DO NOT INCLUDE GRAPHICS IN YOUR MANUSCRIPT
% - Figures should be uploaded separately from your manuscript file.
% - Figures generated using LaTeX should be extracted and removed from the PDF before submission.
% - Figures containing multiple panels/subfigures must be combined into one image file before submission.
% For figure citations, please use "Fig" instead of "Figure".
% See http://journals.plos.org/plosone/s/figures for PLOS figure guidelines.
%
% Tables should be cell-based and may not contain:
% - spacing/line breaks within cells to alter layout or alignment
% - do not nest tabular environments (no tabular environments within tabular environments)
% - no graphics or colored text (cell background color/shading OK)
% See http://journals.plos.org/plosone/s/tables for table guidelines.
%
% For tables that exceed the width of the text column, use the adjustwidth environment as illustrated in the example table in text below.
%
% % % % % % % % % % % % % % % % % % % % % % % %
%
% -- EQUATIONS, MATH SYMBOLS, SUBSCRIPTS, AND SUPERSCRIPTS
%
% IMPORTANT
% Below are a few tips to help format your equations and other special characters according to our specifications. For more tips to help reduce the possibility of formatting errors during conversion, please see our LaTeX guidelines at http://journals.plos.org/plosone/s/latex
%
% For inline equations, please be sure to include all portions of an equation in the math environment.
%
% Do not include text that is not math in the math environment.
%
% Please add line breaks to long display equations when possible in order to fit size of the column.
%
% For inline equations, please do not include punctuation (commas, etc) within the math environment unless this is part of the equation.
%
% When adding superscript or subscripts outside of brackets/braces, please group using {}.
%
% Do not use \cal for caligraphic font.  Instead, use \mathcal{}
%
% % % % % % % % % % % % % % % % % % % % % % % %
%
% Please contact latex@plos.org with any questions.
%
% % % % % % % % % % % % % % % % % % % % % % % %

\documentclass[10pt,letterpaper]{article}
\usepackage[top=0.85in,left=2.75in,footskip=0.75in]{geometry}

% amsmath and amssymb packages, useful for mathematical formulas and symbols
\usepackage{amsmath,amssymb}

% Use adjustwidth environment to exceed column width (see example table in text)
\usepackage{changepage}

% Use Unicode characters when possible
\usepackage[utf8x]{inputenc}

% textcomp package and marvosym package for additional characters
\usepackage{textcomp,marvosym}

% cite package, to clean up citations in the main text. Do not remove.
% \usepackage{cite}

% Use nameref to cite supporting information files (see Supporting Information section for more info)
\usepackage{nameref,hyperref}

% line numbers
\usepackage[right]{lineno}

% ligatures disabled
\usepackage{microtype}
\DisableLigatures[f]{encoding = *, family = * }

% color can be used to apply background shading to table cells only
\usepackage[table]{xcolor}

% array package and thick rules for tables
\usepackage{array}

% create "+" rule type for thick vertical lines
\newcolumntype{+}{!{\vrule width 2pt}}

% create \thickcline for thick horizontal lines of variable length
\newlength\savedwidth
\newcommand\thickcline[1]{%
  \noalign{\global\savedwidth\arrayrulewidth\global\arrayrulewidth 2pt}%
  \cline{#1}%
  \noalign{\vskip\arrayrulewidth}%
  \noalign{\global\arrayrulewidth\savedwidth}%
}

% \thickhline command for thick horizontal lines that span the table
\newcommand\thickhline{\noalign{\global\savedwidth\arrayrulewidth\global\arrayrulewidth 2pt}%
\hline
\noalign{\global\arrayrulewidth\savedwidth}}


% Remove comment for double spacing
%\usepackage{setspace}
%\doublespacing

% Text layout
\raggedright
\setlength{\parindent}{0.5cm}
\textwidth 5.25in
\textheight 8.75in

% Bold the 'Figure #' in the caption and separate it from the title/caption with a period
% Captions will be left justified
\usepackage[aboveskip=1pt,labelfont=bf,labelsep=period,justification=raggedright,singlelinecheck=off]{caption}
\renewcommand{\figurename}{Fig}

% Use the PLoS provided BiBTeX style
% \bibliographystyle{plos2015}

% Remove brackets from numbering in List of References
\makeatletter
\renewcommand{\@biblabel}[1]{\quad#1.}
\makeatother



% Header and Footer with logo
\usepackage{lastpage,fancyhdr,graphicx}
\usepackage{epstopdf}
%\pagestyle{myheadings}
\pagestyle{fancy}
\fancyhf{}
%\setlength{\headheight}{27.023pt}
%\lhead{\includegraphics[width=2.0in]{PLOS-submission.eps}}
\rfoot{\thepage/\pageref{LastPage}}
\renewcommand{\headrulewidth}{0pt}
\renewcommand{\footrule}{\hrule height 2pt \vspace{2mm}}
\fancyheadoffset[L]{2.25in}
\fancyfootoffset[L]{2.25in}
\lfoot{\today}

%% Include all macros below

\newcommand{\lorem}{{\bf LOREM}}
\newcommand{\ipsum}{{\bf IPSUM}}


% Pandoc citation processing
\newlength{\csllabelwidth}
\setlength{\csllabelwidth}{3em}
\newlength{\cslhangindent}
\setlength{\cslhangindent}{1.5em}
% for Pandoc 2.8 to 2.10.1
\newenvironment{cslreferences}%
  {}%
  {\par}
% For Pandoc 2.11+
\newenvironment{CSLReferences}[3] % #1 hanging-ident, #2 entry sp
 {% don't indent paragraphs
  \setlength{\parindent}{0pt}
  % turn on hanging indent if param 1 is 1
  \ifodd #1 \everypar{\setlength{\hangindent}{\cslhangindent}}\ignorespaces\fi
  % set line spacing
  % set entry spacing
  \ifnum #2 > 0
  \setlength{\parskip}{#3\baselineskip}
  \fi
 }%
 {}
\usepackage{calc} % for \widthof, \maxof
\newcommand{\CSLBlock}[1]{#1\hfill\break}
\newcommand{\CSLLeftMargin}[1]{\parbox[t]{\maxof{\widthof{#1}}{\csllabelwidth}}{#1}}
\newcommand{\CSLRightInline}[1]{\parbox[t]{\linewidth}{#1}}
\newcommand{\CSLIndent}[1]{\hspace{\cslhangindent}#1}




\usepackage{forarray}
\usepackage{xstring}
\newcommand{\getIndex}[2]{
  \ForEach{,}{\IfEq{#1}{\thislevelitem}{\number\thislevelcount\ExitForEach}{}}{#2}
}

\setcounter{secnumdepth}{0}

\newcommand{\getAff}[1]{
  \getIndex{#1}{HPME, University of Toronto,Science Faculty, UNAM.}
}

\providecommand{\tightlist}{%
  \setlength{\itemsep}{0pt}\setlength{\parskip}{0pt}}

\begin{document}
\vspace*{0.2in}

% Title must be 250 characters or less.
\begin{flushleft}
{\Large
\textbf\newline{Multi-level multi-state modelling applied to hospital
admission in mexican patients with
COVID-19} % Please use "sentence case" for title and headings (capitalize only the first word in a title (or heading), the first word in a subtitle (or subheading), and any proper nouns).
}
\newline
% Insert author names, affiliations and corresponding author email (do not include titles, positions, or degrees).
\\
Diaz Martinez Juan\textsuperscript{\getAff{HPME, University of
Toronto.}}\textsuperscript{*},
Fuentes-Garc\'{\i}a Ruth\textsuperscript{\getAff{Science Faculty,
UNAM.}}\textsuperscript{*}\\
\bigskip
\textbf{\getAff{HPME, University of Toronto}}Department 1, Street, City,
State, Zip\\
\textbf{\getAff{Science Faculty, UNAM.}}Mathematics Department 2, Mexico City,  04510\\
\bigskip
* Corresponding author: juan.diaz.martinez@mail.utoronto.ca\\
* Corresponding author: rfuentes@ciencias.unam.mx\\
\end{flushleft}
% Please keep the abstract below 300 words
\section*{Abstract}
Lorem ipsum dolor sit amet, consectetur adipiscing elit. Curabitur eget
porta erat. Morbi consectetur est vel gravida pretium. Suspendisse ut
dui eu ante cursus gravida non sed sem. Nullam sapien tellus, commodo id
velit id, eleifend volutpat quam. Phasellus mauris velit, dapibus
finibus elementum vel, pulvinar non tellus. Nunc pellentesque pretium
diam, quis maximus dolor faucibus id. Nunc convallis sodales ante, ut
ullamcorper est egestas vitae. Nam sit amet enim ultrices, ultrices elit
pulvinar, volutpat risus.

% Please keep the Author Summary between 150 and 200 words
% Use first person. PLOS ONE authors please skip this step.
% Author Summary not valid for PLOS ONE submissions.
%\section*{Author summary}
%Lorem ipsum dolor sit amet, consectetur adipiscing elit. Curabitur eget
%porta erat. Morbi consectetur est vel gravida pretium. Suspendisse ut
%dui eu ante cursus gravida non sed sem. Nullam sapien tellus, commodo id
%velit id, eleifend volutpat quam. Phasellus mauris velit, dapibus
%finibus elementum vel, pulvinar non tellus. Nunc pellentesque pretium
%diam, quis maximus dolor faucibus id. Nunc convallis sodales ante, ut
%ullamcorper est egestas vitae. Nam sit amet enim ultrices, ultrices elit
%pulvinar, volutpat risus.

\linenumbers

% Use "Eq" instead of "Equation" for equation citations.
\newcommand{\N}{\mathbb{N}}
\newcommand{\Z}{\mathbb{Z}}
\newcommand{\R}{\mathbb{R}}
\newcommand{\Q}{\mathbb{Q}}
\newcommand{\vac}{\varnothing}
\newcommand{\Pro}{\mathbb{P}}
\newcommand{\var}{\text{Var}}
\newcommand{\E}{\mathbb{E}}

\hypertarget{introduction}{%
\section{Introduction}\label{introduction}}

The severe acute respiratory syndrome coronavirus 2 (SARS-CoV-2) pandemic was declared a Public Health Emergency of
International Concern on January 30, 2020 by the World Health
Organization.The Mexican Health Authorithies declared the first lockdown
on March 26 with 585 cases and 8 deaths reported for COVID-19 {[}2{]};
by the end of the lockdown (June 5th, 2020) total number of cases and deaths were 110,026 and 13,170, respectively. By November 1, Mexico became the fourth country in
number of deaths of SARS-CoV-19 (106,765 deaths), with 1,122,362 incident
cases {[}3{]}; by April 15th  2021 the number of deaths had raised to 214,372 with 2,309,099 incident cases. 
 
Over time it has become clear that comorbidy factors such as
hypertension, diabetes, obesity and smoking increase
the seriousness of the disease, leading to a higher rate of hospitalizations with an additional 25\% of the
cases requiring intensive care unit (ICU) admission and ultimately, intubation {[}4{]}. Mexico ranks second in obesity among OECD countries, with an obesity rate of 72.5\%  among the adult population, which is associated
with the high prevalence of type 2 diabetes, estimated at 13\% of the
adult population in 2017, the highest rate among OECD
countries; the rate of hypertension is also one of the highest chronic diseases among
adult population with 30\% {[}5{]}. The high prevalence of these comorbidities
besides the precarious health care system could be among the main 
main reasons of the elevated severity of the number of cases and deaths rates in the country.  

In Mexico the health institutions are divided in public and private services. There are different public institutions which provide care to different sets of the population: the state employees, the army and naval members, the oil state company (PEMEX) employees and private companies employees. There are also public hospitals for population with no health service coverage. In general, the care withing different healthcare providers cannot be considered homogeneous, therefore it should be relevant for the final outcome of a COVID-19  patient.  

There have been  efforts to understand how patients with
comorbilities are affected by  COVID-19; the work by {[}8{]} proposed a clinical score to
predict COVID-19 lethality, including different factors like
diabetes and obesity among mexican population. This work lead to believe that
obesity mediates 49.5\% of the effect of diabetes on COVID-19 lethality.
Early-onset diabetes conferred an increased risk of hospitalization while
obesity increased the risk ICU for admission
and intubation.

After onset of infection there is a period of time between symptom
detection and hospitalization. The time elapsed before patients approach
hospitals could be excessively long. Once patients are admitted to
hospital, there is also a period of time between the admission and
death. Estimation of these lengths of time through a multilevel model
could enable a better information system to estimate incidence and
transmission rates, particularly at regional level where differences
can be apparent. 

This work considers a multi-state model under a Bayesian framework to
estimate times between symptom detection and hospitalization and between
hospitalization and death. Data used in the modeling comes from the
official database by the Mexican Ministry of Health; the analysis
provides of general overview of hospitalizations in each state of the
country and the different health institutions within. Variables
affecting the patient's final outcome such as the aforementioned
comorbidites are included in the model. Additionally regional
heterogeneity is accounted for trough nested models that consider the
regional contribution and also the health service provider. Other
efforts in recent literature {[}12{]} \textcolor{red}{ME PARECE QUE ESTA CITA NO APARECE} have considered more states
(hospitalization-ICU, ICU-death, ICU-discharged), which allows researchers to asses whether improvements in patient outcomes have been sustained, finding evidence that median hospital stays have lengthened.
Unfortunately the data available for Mexico lacks the necessary granularity to determine such states. Nevertheless, we believe this model
could better inform the estimation of the  incidence and
transmission rates, which is  particularly important while new variants and
increased transmission rates are present.



\hypertarget{methods-and-materials}{%
\section{Methods and materials}\label{methods-and-materials}}

\hypertarget{data-source-and-study-population}{%
\subsection{Data Source and Study
Population}\label{data-source-and-study-population}}

We conducted a prevalence study of the official database from the Mexican
Ministry of Health, these data provide an overview of hospital
admissions, deaths and the period of time between hospitalizations and
first COVID-19 symptoms between March and December 2020. The data analyzed
included mexican adult population diagnosed with COVID-19 in the whole
country. Exclusion criteria were the observations with incomplete
data about hospital admission, symptoms or comorbidities. Additionally,
patients whose time of initial symptoms was captured as the day they were
admitted to hospital were removed, since this time was likely to be
unknown. After applying exclusion criteria a total sample of 1200
registers of adult patients belonging to any healthcare provider,
either private or public in the 32 states of Mexico was selected,
preserving the population characteristics. \textcolor{red}{JUAN PABLO ESTE NUMERO DE
MUESTRA ES CORRECTO?}

Comorbidities that could worsen the patient outcome, affecting the times
we aim to model; such as diabetes, hypertension, obesity, chronic
obstructive pulmonary disease (COPD), asthma, immunosuppression and
chronic kidney disease were included as linear predictors for each state
in the model and so two different relevant groups of variables were statistically
significant for each state:
\[
\begin{aligned}
x_{death} &= (\sim Diabetes+COPD+Obesity \\
 & +Hypertension+Diabetes*Obesity*Hypertension\\
 &+Kidney\_Disease) \\
x_{hospitalization} &= (\sim COPD+Obesity+Kidney\_Disease \\
&+Asthma+Immunosuppression)
\end{aligned}
\]

About 87\% of the population in Mexico is affiliated to some healthcare
provider, but during this pandemic the mexican government has established
a list of hospitals designated to treat COVID-19 patients without any
affiliation distinction. In this study we identified 6 different
healthcare providers which were classified according to their sectors IMSS, ISSSTE,
SEDENA/SEMAR/PEMEX, SSA, ESTATALES (healthcare provider within each
state) these 5 are public care providers while the sixth sector is
private hospitals. It is worth mentioning that following the national
hospital transformation plan {[}6{]}, IMSS alone has transformed about
260 medical units to treat COVID-19 patients {[}9{]}.

\hypertarget{modelling}{%
\subsection{Modelling}\label{modelling}}

We developed four different Bayesian models for trajectories of interest
namely, \emph{Symptoms-Hospitalization} and
\emph{Hospitalization-Death}, in which non-informative initial
distributions were used, located near \(0\), to improve convergence.
\\


\textbf{Figure 1}. Graphical representation of the model: directed acyclic grph (DAG).


Additionally a \textit{QR} reparameterization for the covariate matrix
was used, that is, if \(X\) is an \(n \times m\) covariate matrix,
corresponding to the aforementioned comorbidities, \(X=QR\), where \(Q\)
is an ortogonal matrix and \(R\) is an upper triangular matrix. In
practice, considering \(X=Q'R'\) where \(Q'=Q\sqrt{n-1}\) and
\(R'=\frac{1}{\sqrt{n-1}}R\) is convenient. Hence if \(\zeta\) is the
\textit{N} linear predictor vector such that \(\zeta=X\beta\), with
\(\beta\) an \textit{M} coefficient vector, then
\(\zeta=X\beta=QR\beta=Q'R'\beta\). We used \(\zeta=Q'R'\beta\) for
numerical stability.

Each model captures different levels of information, as more levels were included it
was possible to differentiate the results according to the added information. The four levels  were:




\hypertarget{model-i-one-level}{%
\subsubsection{Model I: One level}\label{model-i-one-level}}

Vectors \(M\) and \(H\) correspond to  survival times 
for deaths and hospitalizations, respectively. We considered a different
set of covariables in each case and assumed deaths and
hospitalizations are independent, we then define the model as

\[
\begin{aligned}
 {M}  &\sim Weibull(\alpha,\mathbf{\eta})\\
 {H}  &\sim Weibull(\alpha,\mathbf{\upsilon}) \\
 \mathbf{\eta} &= \exp\left(-\frac{\mu_m+\mathbf{Q}^*\mathbf{\vartheta}}{\alpha}\right) \\
 \mathbf{\upsilon} &= \exp\left(-\frac{\mu_h+\mathbf{Q}^{**}\mathbf{\theta}}{\alpha}\right) \\
 \alpha&=\exp(\alpha_r*\tau_\alpha) \\
 \alpha_r&\sim N(0,1) \\
 \mu_m,\mu_h &\sim N(0,\tau_\mu) \\
 \mathbf{\vartheta},\mathbf{\theta} &\sim U(-\infty,\infty) \\
\end{aligned}
\]

where \(Q^*\) and \(Q^{**}\) are matrices of standarized covariables for
deaths and hospitalizations respectively and \(\tau_\alpha\) and
\(\tau_{\mu}\) are given positive values \textcolor{red}{QUIZA AQUI PONER LOS VALORES}. This model is described in
grey in Figure 1.

\hypertarget{model-ii-two-levels}{%
\subsubsection{Model II: two levels}\label{model-ii-two-levels}}

The second model is based on the first one, an additional level is added
to account for each state of Mexico to model deaths. The hospitalization
\(H\) remains unchanged and for each state \(l= 1,\ldots, 32\), deaths are summarized in matrix \(M_{l}\) the model is
defined as

\[
\begin{aligned}
 M_l   &\sim Weibull(\alpha,\mathbf{\eta})\\
 H  &\sim Weibull(\alpha,\mathbf{\upsilon}) \\
 \mathbf{\eta} &= \exp\left(-\frac{\mu_m+\mathbf{\mu}_l^r+\mathbf{Q}^*\mathbf{\vartheta}}{\alpha}\right),\space\space l=1,...,32\\
 \mathbf{\upsilon} &= \exp(-\frac{\mu_h+\mathbf{Q}^{**}\mathbf{\theta}}{\alpha}) \\
 \mathbf{\mu}_l&=\sigma*\mathbf{\mu}_l^r \\
 \alpha&=\exp(\alpha^r*\tau_\alpha) \\
 \alpha^r&\sim N(0,1) \\
 \mu^r_l&\sim N(0,1) \\
 \sigma&\sim t^+_3(0,1) \\
 \mu_m,\mu_h &\sim N(0,\tau_\mu) \\
 \mathbf{\vartheta},\mathbf{\theta} &\sim U(-\infty,\infty) \\
\end{aligned}
\] where \(Q^*\) and \(Q^{**}\) are matrices of standarized covariables
for deaths and hospitalizations respectively and \(\tau_\alpha\) and
\(\tau_{\mu}\) are given positive values. This model is described in
green in Figure 1.

\hypertarget{model-iii-three-levels}{%
\paragraph{Model III: Three levels}\label{model-iii-three-levels}}

Based on Model II, we consider a third level to include the healthcare provider where patients are hospitalized, \(k\), for patient \(i\)
in state \(l\) we have \(M_{l,k}\) for the corresponding death matrix
and the model is given as

\[
\begin{aligned}
 M_{l,k}   &\sim Weibull(\alpha,\mathbf{\eta})\\
 H  &\sim Weibull(\alpha,\mathbf{\upsilon}) \\
 \mathbf{\eta} &= \exp \left(-\frac{\mu_m+\mathbf{\mu}_l^r+\mathbf{\mu}_k^r+\mathbf{Q}^*\mathbf{\vartheta}}{\alpha} \right),\space\space l=1,...,32,\space k=1,...,5\\
 \mathbf{\upsilon} &= \exp\left(-\frac{\mu_h+\mathbf{Q}^{**}\mathbf{\theta}}{\alpha}\right) \\
  \mathbf{\mu}_l&=\sigma_l*\mathbf{\mu}_l^r, \space\space l=1,2 \\
 \mathbf{\mu}_k&=\sigma_l*\mathbf{\mu}_k^r \\
 \alpha&=\exp(\alpha^r*\tau_\alpha) \\
 \alpha^r&\sim N(0,1) \\
 \mu^r_l,\mu^r_k &\sim N(0,1) \\
 \sigma_l&\sim t^+_3(0,1) \\
 \mu_m,\mu_h &\sim N(0,\tau_\mu) \\
 \mathbf{\vartheta},\mathbf{\theta} &\sim U(-\infty,\infty) \\
\end{aligned}
\] where \(Q^*\) and \(Q^{**}\) are matrices of standarized covariables
for deaths and hospitalizations respectively and \(\tau_\alpha\) and
\(\tau_{\mu}\) are given positive values. This model is described in
yellow in Figure 1.

\hypertarget{model-iv}{%
\subsubsection{Model IV:}\label{model-iv}}

Based as well on Model II, we consider index \(j=(l,k)\) where
\(l\) accounts for the \(l\)-th state and \(k\) for the healthcare provider, \(j \in \{1,...,153\}\) where the distribution for deaths is given by

\[
\begin{aligned}
 M_j &\sim Weibull\left(\alpha, \mathbf{\eta} \right) \\
 H  &\sim Weibull(\alpha,\mathbf{\upsilon}) \\
 \mathbf{\eta} &= \exp \left(-\frac{\mu_m+\mathbf{\mu}_j^r+\mathbf{Q}^*\mathbf{\vartheta}}{\alpha} \right),\space\\
 \mathbf{\upsilon} &= \exp\left(-\frac{\mu_h+\mathbf{Q}^{**}\mathbf{\theta}}{\alpha}\right) \\
 \mathbf{\mu}_j&=\sigma_j*\mathbf{\mu}_j^r \\
 \alpha&=\exp(\alpha^r*\tau_\alpha) \\
 \alpha^r&\sim N(0,1) \\
 \mu^r_l,\mu^r_k &\sim N(0,1) \\
 \sigma_j&\sim t^+_3(0,1) \\
 \mu_j &\sim N(0,\tau_\mu) \\
 \mathbf{\vartheta},\mathbf{\theta} &\sim U(-\infty,\infty) \\
\end{aligned}
\] where \(Q^*\) and \(Q^{**}\) are matrices of standarized covariables
for deaths and hospitalizations respectively and \(\tau_\alpha\) and
\(\tau_{\mu}\) are given positive values. 

\textcolor{red}{JUAN PABLO, EL ULTIMO MODELO HAY QUE CHECARLO PARA VER QUE SEA CORRECTO.}



To choose the model that best fits the data we considered
the leave-one-out cross-validation (LOO) proposed by {[}1{]}, which
estimates pointwise out-of-sample prediction accuracy, using the
log-likelihood evaluated at the posterior simulations of the parameter
values, Table 1.

\begin{table}[!htb]
\centering
\begin{tabular}{lcc}
\hline
{\textbf{Model}} & {\textbf{elpd leave one out}} & {\textbf{p leave one out}} \\
\hline Model 1  &  -75473.3 (187.8) & 26.9 (2.2) \\
Model II          &   -75352.2 (186.3) &  84.1 (4.7)\\
Model III         &    -75284.9 (186.6) &  92.5 (4.9)  \\
\hline
\end{tabular}
\caption{\label{tab:gof} Expected log-pointwise predictive density (ELPD) for a new data set and effective number of parameters (standard deviation).}
\end{table}



\hypertarget{results}{%
\section{Results}\label{results}}

The parameters were estimated using Stan \textcolor{red}{DAR QUIZA MÁS DETALLES}. We show
results for model III which performed better in terms of the
likelihood and showed good convergence of all parameteres. The posterior
0.95 credibility intervals for parameters of interest at different
levels of the model are shown in Figures 2 and 3. It is worth pointing
out that we are displaying the log hazard ratio, hence positive values
for parameters will point to increasing risks for the corresponding
transition and level.



\textbf{Figure 2}. Hazard rate in logarithmic scale by state, credible interval of 95\%.




\textbf{Figure 3}. Hazard rate in logarithmic scale by health institution, credible interval of 95\%.

Increased risk for hospitalization was observed at the global population
level for chronic renal disease, whereas for death such was the case for
COPD and the interaction of diabetes:hypertension:obesity.


\textbf{Figure 4}. Predictive distribution for deaths.


Our results show that there are differences in mortality between the
states without accounting for institution and it is
related to the prompt time of death or viceversa. Figure 2
shows the states in which the overall rate of mortality is higher such
as Campeche, Colima, Guanajuato, Hidalgo, Jalisco, Morelos, Nayarit,
Oaxaca, Puebla, Tabasco and Veracruz. The difference might be linked to
the late hospitalization of patients.


Figure 4 displays evidence that 5 days after hospitalization
there is a peak on mortality rate, which could be related due the late
hospitalization of patients with mild symptoms who developed ``happy
hypoxemia,'' that is extremely low blood oxygenation, but without
sensation of dyspnea. In Wuhan, within a cohort of patients infected with (SARS-COV-2) who id not present dyspnea 62\% showed
severe disease and 46\%  ended up intubated, ventilated or dead {[}7{]}.

Regarding the 6 healthcare providers  included in the analysis
differences were also found. While state-managed hospitals and private
sector showed lower risks, in contrast the IMSS seems to be the one with
the highest risk (Figure 3). Although it is worth mentioning that
following the national hospital transformation plan {[}6{]}, IMSS alone
has transformed about 260 medical units to treat Covid-19 patients
{[}9{]}. Additionaly it has the largest number of affiliations and they are  liklely to have higher risk exposure. 

\hypertarget{disscusion}{%
\section{Disscusion}\label{disscusion}}

Multiple sources have shownthat the presence of comorbidities such
as diabetes, hypertension, obesity, chronic obstructive pulmonary
disease (COPD), asthma, inmmuno-suppression and chronic kidney disease
are associated to a worse outcome for the patients diagnosed with
COVID-19. Particularly for those who are hospitalized in the ICU due
intubation. To our knowledge this is the first study in Mexico which
analyzed the time elapsed between the patient´s first symptoms,
hospitalization and death; these analyses were further broken down to the
different states of the republic and the healthcare providers within
them. Thus it is possible to identify providers and states with an increased
risk of hospitalization and death. Mexico is the
third place in obesity among the OCDE countries, which could be the main reason of the high number of severe cases of COVID-19.

One problem that aggravates the situation is the precarious state of the public
healthcare system which universal coverage is estimated about 87\% of
the mexican population. It is clear that mexican healthcare has overrun
during this pandemic and has appealed to private health providers to
cope with the treatment of COVID-19 patients. Among all health care
providers the IMSS is the one with the highest risk, however
it is the largest healthcare provider across Mexico with hospitals from
level 2-level 4 of which ``Siglo XXI'' is a country-leader in
research and innovative treatments and procedures. In March, 2020 38  hospitals were ``converted'' {[}6{]} to exclusively treat
COVID-19  among which 18 were IMSS hospitals
only; after one year 960 hospitals across the country were converted to
treat patients with COVID-19 of these 289 (30.10\%) belong to IMSS
{[}9{]}.

The proposed modelling can be helpful to a regional level to improve healthcare
assistance, it could additionally be useful to inform statistical
estimation of parameters for an epidemiological model.



This study has shown that there are differences in mortality between the
different states of the republic; there are states in which the overall
rate of mortality is higher due the late hospitalization of patients
such as Veracruz, Nuevo Leon, San Luis Potosí, Guanajuato, Chiapas and
Mexico City. Breaking down this analysis to state level we found a
higher risk of hospitalization, specifically in Veracruz  which
 has been historically unsteady regarding the public healthcare system in
sectors like IMSS, ISSSTE, SEDENA/SEMAR/PEMEX. The fact that different final outcomes could be related to patient's
late hospitalization, hence suggesting that the average patient waits
until the symptoms are severe to seek professional healthcare, needs to
be further investigated.

One of the  limitations of the study is the reduced number of states
we were able to include in the modls due to the lack of information
regarding dates of disccharged of recovered patient´s after hospitalization.

\hypertarget{references}{%
\section*{References}\label{references}}
\addcontentsline{toc}{section}{References}

\hypertarget{refs}{}
\begin{CSLReferences}{0}{0}
\leavevmode\hypertarget{ref-Vehtari2017}{}%
\CSLLeftMargin{1. }
\CSLRightInline{Vehtari A, Gelman A, Gabry J. {Practical Bayesian model
evaluation using leave-one-out cross-validation and WAIC}. Statistics
and Computing. 2017;27: 1413--1432.
doi:\href{https://doi.org/10.1007/s11222-016-9696-4}{10.1007/s11222-016-9696-4}}

\leavevmode\hypertarget{ref-Carrillo-Vega2020}{}%
\CSLLeftMargin{2. }
\CSLRightInline{Carrillo-Vega MF, Salinas-Escudero G, García-Peña C,
Gutiérrez-Robledo LM, Parra-Rodríguez L. {Early estimation of the risk
factors for hospitalization and mortality by COVID-19 in Mexico}. PLOS
ONE. Public Library of Science; 2020;15: e0238905. Available:
\url{https://doi.org/10.1371/journal.pone.0238905}}

\leavevmode\hypertarget{ref-OrganizacionPanamericanadelaSalud2021}{}%
\CSLLeftMargin{3. }
\CSLRightInline{Organización Panamericana de la Salud O. {Documentos
t{é}cnicos de la OPS - Enfermedad por el Coronavirus (COVID-19)}
{[}Internet{]}. 2021. Available:
\url{https://www.paho.org/es/documentos-tecnicos-ops-enfermedad-por-coronavirus-covid-19}}

\leavevmode\hypertarget{ref-SEDESA2021}{}%
\CSLLeftMargin{4. }
\CSLRightInline{SEDESA. {Base del Sistema Nacional de Vigilancia
Epidemiologica para el seguimiento a posibles casos de COVID-19 en la
Ciudad de M{é}xico} {[}Internet{]}. 2021. Available:
\url{http://sinave.gob.mx/}}

\leavevmode\hypertarget{ref-OCDE2019}{}%
\CSLLeftMargin{5. }
\CSLRightInline{OCDE. {Health at a Glance 2019: OECD Indicators.}
{[}Internet{]}. OCDE; 2019 p. 4. Available:
\url{https://www.oecd.org/mexico/health-at-a-glance-mexico-ES.pdf}}

\leavevmode\hypertarget{ref-Mendoza-Popoca2020}{}%
\CSLLeftMargin{6. }
\CSLRightInline{Mendoza-Popoca CÚ, Suárez-Morales M. {Hospital
reconversion in response to the COVID-19 pandemic}. Revista Mexicana de
Anestesiologia. 2020;43: 151--156.
doi:\href{https://doi.org/10.35366/92875}{10.35366/92875}}

\leavevmode\hypertarget{ref-Guan2020}{}%
\CSLLeftMargin{7. }
\CSLRightInline{Guan W, Ni Z, Hu Y, Liang W, Ou C, He J, et al.
{Clinical Characteristics of Coronavirus Disease 2019 in China}. New
England Journal of Medicine. 2020;382: 1708--1720.
doi:\href{https://doi.org/10.1056/nejmoa2002032}{10.1056/nejmoa2002032}}

\leavevmode\hypertarget{ref-Bello-Chavolla2020}{}%
\CSLLeftMargin{8. }
\CSLRightInline{Bello-Chavolla OY, Bahena-López JP, Antonio-Villa NE,
Vargas-Vázquez A, González-Díaz A, Márquez-Salinas A, et al. {Predicting
Mortality Due to SARS-CoV-2: A Mechanistic Score Relating Obesity and
Diabetes to COVID-19 Outcomes in Mexico}. Journal of Clinical
Endocrinology and Metabolism. 2020;105: 2752--2761.
doi:\href{https://doi.org/10.1210/clinem/dgaa346}{10.1210/clinem/dgaa346}}

\leavevmode\hypertarget{ref-UNAM2021}{}%
\CSLLeftMargin{9. }
\CSLRightInline{UNAM LI de TeIE(iSTARL del I de G de la. {Sistema de
Informaci{ó}n de la Red IRAG} {[}Internet{]}. 2021. Available:
\url{https://www.gits.igg.unam.mx/red-irag-dashboard/reviewHome}}

\end{CSLReferences}

\nolinenumbers


\end{document}
